\documentclass{article}
\begin{document}

	\parbox{20cm}{
	\footnotesize
	\mbox{
	\parbox[b]{8.7 cm}{
	\begin{minipage}[t]{3.5cm}
		The first line of this 3.5cm wide minipage or parbox is
		aligned with the first line of the neighbouring minipage or parbox.
	\end{minipage}
	\hfill
	\begin{minipage}[t]{5cm}
		This 5 cm wide minipage or parbox is poistioned so that its top line is 
		at the same level as that of the box on its left, while its bottom line
		is even with that of the box on the right. The naive notion that this
		arrangment may be achieved with the positioning arguments set to
		t, t and b is incorrect. Why? What would this selection really produce?
	\end{minipage}
	
	\mbox{ }
	}
	\hfill
	\begin{minipage}[b]{3.5cm}
		The true solution involves the nesting of two of the three structures
		in an enclosing minipage, which is then separately aligned with the third
		one.
		\mbox{}
	\end{minipage}
	}
	}

\end{document}