% The [12pt] option increases the font size
\documentclass[titlepage]{article}

% Here we put in the preamble for this exercise
%\pagestyle{myheadings} 
%\markright{Exercises}
%\pagenumbering{Roman}

% This package suppresses the indents and increases space between paragraphs
\usepackage{parskip}

\begin{document}

% Set parameters for the title page
%\title{Exercises}
%\author{Sion P. Stevenson\\Commerce House, Ambleston}
%\date{Haverfordwest, \today}
%\maketitle

\textbf{Albert Einstein} (14 March 1879 -- 18 April 1955) was
a German-born theoretical physicist and philosopher of science. He
developed the general theory of relativity, one of the two pillars
of modern physics (alongside quantum mechanics). He is best known
in popular culture for his mass-energy equivalence formula $E = mc^2$
(which has been dubbed ``the worlds most famous equation''). He received
the 1921 Nobel Prize in Physics ``for his services to theoretical physics,
and especially for his discovery of the law of the photoelectric effect''.
The latter was pivotal in establishing quantum theory.

Near the beginning of his career, Einstein thought that Newtonian
% The \- command tells Latex where the correct place to split the word over a line break is
mechan\-ics was no longer enough to reconcile the laws of classical
mechan\-ics with the laws of the electromagnetic field. This led to the
development of his special theory of relativity. He realised, however, that
the principle of relativity could also be extended to gravitational fields,
and with his subsequent theory of gravitation in 1916, he published a paper
on the general theory of relativity. He continued to deal with problems of
statistical mechanics and quantum theory, which led to his explanations of
particle theory and the motion of molecules. He also investigated the
thermal properties of light which laid the foundation of the photon theory
of light. In 1917, Einstein applied the general theory of relativity to
model the large-scale structure of the universe.

He was visiting the United States when Adolf Hitler came to power in 1933
and, being Jewish, did not go back to Germany, where he had been a
professor at the Berlin Academy of Sciences. He settled in the U.S.,
becoming an American Citizen in 1940. On the eve of World War II, he
endorsed a letter to President Franklin D. Roosevelt alerting him to the
potential development of ``extremely powerful bombs of a new type'' and
recommending that the U.S. begin similar research. This eventually led
to what would become the Manhattan Project. Einstein supported defending
the Allied forces, but largely denounced the idea of using the newly 
discovered nuclear fission as a new weapon. Later, with the British
philosopher Bertrand Russel, Einstein signed the Russel-Einstein 
Manifesto, which highlighted the danger of nuclear weapons. Einstein
was affiliated with the Institute for Advanced Study in Princeton,
New Jersey, until his death in 1955.

Einstein published more than 300 scientific papers along with over
150 non-scientific works. On 5 December 2014 universities and
archives announced the release of Einstein's papers, comprising more
than 30,000 unique documents. Einstein's intellectual achievements and
originality have made the word ``Einstein'' synonymous with genius so
that in a sense he may be regarded as the greatest genius who ever lived.

\section{Life}
\subsection{Early life and education}

Albert Einstein was born in Ulm, in the Kingdom of Wuttemburg in
the German Empire on 14 March 1879. His parents were Hermann Einstein,
a salesman and engineer, and Pauline Koch. In 1880, the family moved
to Munich, where his father and his uncle founded
\emph{Elektrotechnische Fabrik J. Einstein \& Cie}, a company that
manufactured electrical equipment based on direct current.

The Einsteins were non-observant Ashkenazi Jews. Albert attended a
Catholic elementary school from the age of 5 for three years. At the
age of 8, he was transferred to the Luitpold Gymnasium (now known as
the Albert Einstein Gymnasium), where he received advanced primary and
secondary school education until he left Germany seven years later.
Contrary to popular suggestions that he struggled with early speech
difficulties, the Albert Einstein Archives indicate he excelled at
the first school he attended. He was right-handed; there appears to
be no evidence for the widespread popular belief that he was left-handed.

His father once showed him a pocket compass; Einstein realized that
there must be something causing the needle to move, despite the apparent
``empty space''. As he grew, Einstein built models and mechanical devices
for fun and began to show a talent for mathematics.

In 1894, his fathers' company failed: direct current (DC) lost the War of
Currents to alternating current (AC). In search of business, the Einstein
family moved to Italy, first to Milan and then, a few months later, to Pavia.
When the family moved to Pavia, Einstein stayed in Munich to finish his
studies at the Luitpold Gymnasium. His father intended for him to pursue
electrical engineering, but Einstein clashed with authorities and resented
the school's regimen and teaching method. He later wrote that the spirit of
learning and creative thought were lost in strict rote learning. At the end
of December 1894, he travelled to Italy to join his family in Pavia,
convincing the school to let him go by using a doctor's note. It was during
his time in Italy that he wrote a short essay with the title ``On the
investigation of the State of the Ether in a Magnetic Field''.

In 1895, at the age of 16, Einstein sat the entrance examinations for the
Swiss Federal Polytechnic in Z\"{u}rich (later the Eidgen\"{o}ssische
Technische Hochschule ETH). He failed to reach the required standard in
the general part of the examination, but obtained exceptional grades
in physics and mathematics. On the advice of the pricipal of the 
Polytechnic, he attended the Argovian cantonal school (gymnasium) in
Aarau, Switzerland, in 1895--96 to complete his secondary schooling.
While lodging with the family of Professor Jost Winteler, he fell in
love with Winteler's daughter, Marie. (Albert's sister Maja later
married Winteler's son Paul.) In January 1896, with his father's approval,
he renounced his citizenship in the German Kingdom of W\"{u}rttemberg
to avoid military service. In September 1896, he passed the Swiss Matura
with mostly good grades, including a top grade of 6 in physics and 
mathematical subjects, in a scale of 1--6. Though only 17, he enrolled
in the four-year mathematics and physics teaching diploma program at
the Z\"{u}rich Polytechnic. Marue Winteler moved to Olsberg, Switzerland
for a teaching post.

Einstein's future wife, Mileva Mari\'{c}, also enrolled at the Polytechnic
that same year. She was the only woman among the six students in the
mathematics and physics section of the teaching diploma course. Over
the next few years, Einstein and Mari\'{c}'s friendship developed into
romance. and they read books together on extra-curricular physics in
which Einstein was taking an increasing interest. In 1900, Einstein was
awarded the Z\"{u}rich Polytechnic teaching diploma, but Mari\'{c} failed
the examination with a poor grade in the mathematics component, theory of
functions. There have been claims that Mari\'{c} collaborated with Einstein
on his celebrated 1905 papers, but historians of physics who have studied
the issue find no evidence that she made any substantive contributions.

\subsection{Marriages and children}

The discovery and publication in 1987 of an early correspondence between
Einstein and Mari\'{c} revealed that they had had a daughter, called
``Lieserl'' in their letters, born in early 1902 in Novi Sad where
Mari\'{c} was staying with here parents. Mari\'{c} returned to
Switzerland without the child, whose real name and fate are unknown.
Einstein probably never saw his daughter. The contents of his letter
to Mari\'{c} in September 1903 suggest that the girl was either adopted
or died of scarlet fever in infancy.

Einstein and Mari\'{c} married in January 1903. In May 1904, the couple's
first son Hans Albert Einstein, was born in Bern, Switzerland. Their
second son, Eduard, was born in Zurich in July 1910. In 1914, the couple
separated; Einstein moved to Berlin and his wife remained in Zurich
with their sons. They divorced on 14 February 1919, having lived apart
for five years. Eduard, whom is father called ``Tete'' (for \emph{petit},
had a breakdown at about age 20 and was diagnosed with schizophrenia. His
mother cared for him and he was also committed to asylums for several
periods, including full-time after her death.

Einstein married Elsa L\"{o}wenthal on 2 June 1919, after having had
a relationship with her since 1912. She was a first cousin maternally
and a second cousin paternally. In 1933, they emigrated to the to the
United States. In 1935, Elsa Einstein was diagnosed with heart and 
kidney problems; she died in December 1936.

\subsection{Patent office}

After graduating, Einstein spent almost two frustrating years searching
for a teaching post. He acquired Swiss citizenship in February 1901,
but was not conscripted for medical reasons. With the help of Marcel
Grossmann's father Einstein secured a job in Bern at the Federal Office
for Intellectual Property, the patent office, as an assisstant examiner.
He evaluated patent applications for a variety of devices including
a gravel sorter and an electromechanical typewriter. In 1903, Einstein's
position at the Swiss Patent Office became permanent, although he
was passed over for promotion until he ``fully mastered machine
technology''.

\section{Scientific career}

Throughout his life, Einstein published hundreds of books and articles.
In addition to the work he did by himself he also collaborated with
other scientific on additional projects including the Bose-Einstein
statistics, the Einstein refrigerator and more.

\subsection{Thermodynamic fluctuations and statistical physics}

Albert Einstein's first paper submitted in 1900 to 
\emph{Annalen der Physik} was on capillary attraction. It was published
in 1901 with the title ``Folgerungen aus den
Capillarit\"{a}tserscheinungen'', which translates as ``Conclusions
from the capillarity phenomena''. Two papers he published in
1902--1903 (thermodynamics) attempted to interpret atomic phenomena
from a statistical point of view. These papers were the foundation
for the 1905 paper on Brownian motion, which showed that Brownian
movement can be construed as firm evidence that molecules exist. His
research in 1903 and 1904 was mainly concerned with the effect of
finite atomic size on diffusion phenomena.

\end{document}