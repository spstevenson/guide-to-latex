\documentclass{article}
\begin{document}
\textbf{Albert Einstein} (14 March 1879 -- 18 April 1955) was
a German-born theoretical physicist and philosopher of science. He
developed the general theory of relativity, one of the two pillars
of modern physics (alongside quantum mechanics). He is best known
in popular culture for his mass-energy equivalence formula $E = mc^2$
(which has been dubbed ``the worlds most famous equation''). He received
the 1921 Nobel Prize in Physics ``for his services to theoretical physics,
and especially for his discovery of the law of the photoelectric effect''.
The latter was pivotal in establishing quantum theory.

Near the beginning of his career, Einstein thought that Newtonian
mechanics was no longer enough to reconcile the laws of classical
mechanics with the laws of the electromagnetic field. This led to the
development of his special theory of relativity. He realised, however, that
the principle of relativity could also be extended to gravitational fields,
and with his subsequent theory of gravitation in 1916, he published a paper
on the general theory of relativity. He continued to deal with problems of
statistical mechanics and quantum theory, which led to his explanations of
particle theory and the motion of molecules. He also investigated the
thermal properties of light which laid the foundation of the photon theory
of light. In 1917, Einstein applied the general theory of relativity to
model the large-scale structure of the universe.

He was visiting the United States when Adolf Hitler came to power in 1933
and, being Jewish, did not go back to Germany, where he had been a
professor at the Berlin Academy of Sciences. He settled in the U.S.,
becoming an American Citizen in 1940. On the eve of World War II, he
endorsed a letter to President Franklin D. Roosevelt alerting him to the
potential development of ``extremely powerful bombs of a new type'' and
recommending that the U.S. begin similar research. This eventually led
to what would become the Manhattan Project. Einstein supported defending
the Allied forces, but largely denounced the idea of using the newly 
discovered nuclear fission as a new weapon. Later, with the British
philosopher Bertrand Russel, Einstein signed the Russel-Einstein 
Manifesto, which highlighted the danger of nuclear weapons. Einstein
was affiliated with the Institute for Advanced Study in Princeton,
New Jersey, until his death in 1955.

Einstein published more than 300 scientific papers along with over
150 non-scientific works. On 5 December 2014m universities and
archives announced the release of Einstein's papers, comprising more
than 30,000 unique documents. Einstein's intellectual achievements and
originality have made the word ``Einstein'' synonymous with genius so
that in a sense he may be regarded as the greatest genius who ever lived.

\end{document}